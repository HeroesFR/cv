%-------------------------------------------------------------------------------
%	SECTION TITLE
%-------------------------------------------------------------------------------
\cvsection{Projets Personnels \& Universitaires}


%-------------------------------------------------------------------------------
%	CONTENT
%-------------------------------------------------------------------------------
\begin{cventries}

%---------------------------------------------------------
  \cventry
	{Software Application using eye control for consumer devices} % Job title
	{EyeTracker} % Nom
	{INSA Rouen Normandie(76)} % Location
	{2016} % Date(s)
	{
		\begin{cvitems} % Description(s) of tasks/responsibilities
			\item{Software Development (\textbf{Java}) the goal was to track the glance with the help of a perception tool (\textit{TheEyeTribe}) }
			%\item {Développement d'une Application \textbf{Java} dans le but de tracer le regard à l'aide d'un outil de perception de regard \textit{EyeTribe}}
		\end{cvitems}
	}

%---------------------------------------------------------
  \cventry
	{Mobile Application for the French Gendarmerie National} % Job title
	{SOS Project} % Nom
	{INSA Rouen Normandie(76)} % Location
	{2016} % Date(s)
	{
		\begin{cvitems} % Description(s) of tasks/responsibilities
			\item{Develoment of a mobile application in Android (\textbf{JAVA} \& \textbf{XML}), the goal was to create an app for tourists in France. The app allows the tourists to call emergency in 2 clicks, having complementary informations depending on the situation (Develop with \textbf{Android Studio}).}
			%\item {Développement d'une Application Android (\textbf{JAVA} \& \textbf{XML}), le but de ce projet était de réaliser une application pour les touristes effectuant leurs voyages en France. L'application permet aux touristes d'appeler les secours en cas de problème en quelques clics , et d'avoir des informations complémentaires en fonction de la situation. (Utilisation d'\textbf{Android Studio}). }
		\end{cvitems}
	}

%---------------------------------------------------------
  \cventry
	{Development of the Othello game and play versus an AI} % Job title
	{Othello} % Nom
	{INSA Rouen Normandie(76)} % Location
	{2015} % Date(s)
	{
		\begin{cvitems} % Description(s) of tasks/responsibilities
			\item{Developed in \textbf{C} using the min-max / alpha-beta algorithm to calculate AI' score.}
			%\item {Développement du jeu réalisé en \textbf{C}, avec l'utilisation de l'algorithme Min-Max pour le calcul du score de l'IA.}
		\end{cvitems}
	}

%---------------------------------------------------------
  \cventry
	{Ruzzle Solver Development} % Job title
	{Ruzzle Solveur} % Nom
	{INSA Rouen Normandie(76)} % Location
	{2014} % Date(s)
	{
		\begin{cvitems} % Description(s) of tasks/responsibilities
			\item {Developed in \textbf{C}, the goal was on one hand to create a binary tree of the french dictonnary (around 320 000 words), serialize this tree and on the other part find every words of the Ruzzle. Then we had to compare the tree to the list of words we found to finally made the best score possible.}
			%\item {Développement du jeu réalisé en \textbf{C}, le but était d'une part de réaliser un arbre binaire de mots du dictionnaire de la langue française (320 000 mots), de le sérialiser, puis de comparer cet arbre aux différents mots du Ruzzle pour obtenir les mots qui rapportent le plus de points.}
		\end{cvitems}
	}

%---------------------------------------------------------
\end{cventries}
