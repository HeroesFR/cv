%-------------------------------------------------------------------------------
%	SECTION TITLE
%-------------------------------------------------------------------------------
\cvsection{Projets Personnels \& Universitaires}


%-------------------------------------------------------------------------------
%	CONTENT
%-------------------------------------------------------------------------------
\begin{cventries}

%---------------------------------------------------------
  \cventry
	{Développement d'un logiciel utilisant un Tracker de regard} % Job title
	{EyeTracker} % Nom
	{INSA Rouen Normandie(76)} % Location
	{2016} % Date(s)
	{
		\begin{cvitems} % Description(s) of tasks/responsibilities
			\item {Développement d'une Application \textbf{Java} dans le but de tracer le regard à l'aide d'un outil de perception de regard \textit{TheEyeTribe}}
		\end{cvitems}
	}

%---------------------------------------------------------
  \cventry
	{Développement d'une Application mobile pour la Gendarmerie Nationale} % Job title
	{Projet SOS} % Nom
	{INSA Rouen Normandie(76)} % Location
	{2016} % Date(s)
	{
		\begin{cvitems} % Description(s) of tasks/responsibilities
			\item {Développement d'une Application Android (\textbf{Java} \& \textbf{XML}), le but de ce projet était de réaliser une application pour les touristes effectuant leurs voyages en France. L'application permet aux touristes d'appeler les secours en cas de problème en quelques clics , et d'avoir des informations complémentaires en fonction de la situation. (Utilisation d'\textbf{Android Studio}). }
		\end{cvitems}
	}

%---------------------------------------------------------
  \cventry
	{Développement du jeu de l'Othello avec la possibilité de jouer contre une Intelligence Artificielle} % Job title
	{Othello} % Nom
	{INSA Rouen Normandie(76)} % Location
	{2015} % Date(s)
	{
		\begin{cvitems} % Description(s) of tasks/responsibilities
			\item {Développement du jeu réalisé en \textbf{C}, avec l'utilisation de l'algorithme Min-Max pour le calcul du score de l'IA.}
		\end{cvitems}
	}

%---------------------------------------------------------
  \cventry
	{Développement d'un solveur de Ruzzle} % Job title
	{Ruzzle Solveur} % Nom
	{INSA Rouen Normandie(76)} % Location
	{2014} % Date(s)
	{
		\begin{cvitems} % Description(s) of tasks/responsibilities
			\item {Développement du jeu réalisé en \textbf{C}, le but était d'une part de réaliser un arbre binaire de mots du dictionnaire de la langue française (320 000 mots), de le sérialiser, puis de comparer cet arbre aux différents mots du Ruzzle pour obtenir les mots qui rapportent le plus de points.}
		\end{cvitems}
	}

%---------------------------------------------------------
\end{cventries}
